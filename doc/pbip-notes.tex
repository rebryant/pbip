\documentclass{easychair}
%\documentclass[runningheads]{llncs}

\usepackage{latexsym}
%\usepackage{times}
\usepackage{amsmath}
\usepackage{stmaryrd}
\usepackage{graphicx}
\usepackage{pict2e}
\usepackage{tikz}
%%\usepackage{pgfplots}
%%\pgfplotsset{compat=1.14}
\usepackage{cite}
\usepackage{booktabs}
\usepackage{adjustbox}
\usepackage{marvosym}
%\usepackage{hyperref}
%% Colored hyperlink 
\newcommand{\cref}[2]{\href{#1}{\color{blue}#2}}
%% Colored hyperlink showing link in TT font
% \newcommand{\chref}[1]{\href{#1}{\small\tt \color{blue}#1}}
\newcommand{\hcref}[1]{\cref{#1}{\small\tt #1}}
%% Change tt font
\usepackage{inconsolata}
\usepackage[T1]{fontenc}
\usepackage{alltt}
%% ccode- for displaying formatted C code (c2tex) 
\newenvironment{ccode}%
{\begin{alltt}}%
{\end{alltt}}

\bibliographystyle{abbrv}
%\bibliographystyle{splncs04}

%% Putting labels in pictures
\newcommand{\booland}{\land}
\newcommand{\boolor}{\lor}
\newcommand{\boolxor}{\oplus}
\newcommand{\boolnot}{\neg}
\newcommand{\tautology}{\top}
\newcommand{\nil}{\bot}
\renewcommand{\obar}[1]{\overline{#1}}
\newcommand{\lit}{\ell}
\newcommand{\ite}{\mbox{\it ITE}}
\newcommand{\restrict}[2]{#1|_{#2}}
\newcommand{\trust}[1]{\dot {#1}}
\newcommand{\assign}{\rho}
\newcommand{\nassign}{\obar{\assign}}
\newcommand{\indices}{I}
\newcommand{\uprop}[1]{\beta[#1]}
\newcommand{\imply}{\Rightarrow}


\newcommand{\opname}[1]{\mbox{\sc #1}}
\newcommand{\andop}{\opname{And}}
\newcommand{\implyop}{\opname{Imply}}

\newcommand{\applyand}{\opname{ApplyAnd}}
\newcommand{\applyor}{\opname{ApplyOr}}
\newcommand{\applyimply}{\opname{ProveImplication}}

\newcommand{\turnstile}{\models}
\newcommand{\func}[1]{\llbracket#1\rrbracket}

\newcommand{\fname}[1]{\mbox{\small\sf #1}}

\newcommand{\lo}{\fname{Lo}}
\newcommand{\hi}{\fname{Hi}}
\newcommand{\var}{\fname{Var}}
\newcommand{\val}{\fname{Val}}
\newcommand{\xvar}[1]{#1}
\newcommand{\node}[1]{\mathbf{#1}}

% Code formatting
\newcommand{\showcomment}[1]{\texttt{/*} \textit{#1} \texttt{*/}}
\newcommand{\keyword}[1]{\textbf{#1}}
\newcommand{\keyif}{\keyword{if}}
\newcommand{\keyfor}{\keyword{for}}
\newcommand{\keyelse}{\keyword{else}}
\newcommand{\keyor}{\keyword{or}}
\newcommand{\keyreturn}{\keyword{return}}
%\newcommand{\assign}{\ensuremath{\longleftarrow}}

\definecolor{redorange}{rgb}{0.878431, 0.235294, 0.192157}
\definecolor{lightblue}{rgb}{0.552941, 0.72549, 0.792157}
\definecolor{clearyellow}{rgb}{0.964706, 0.745098, 0}
\definecolor{clearorange}{rgb}{0.917647, 0.462745, 0}
\definecolor{mildgray}{rgb}{0.54902, 0.509804, 0.47451}
\definecolor{softblue}{rgb}{0.643137, 0.858824, 0.909804}
\definecolor{bluegray}{rgb}{0.141176, 0.313725, 0.603922}
\definecolor{lightgreen}{rgb}{0.709804, 0.741176, 0}
\definecolor{darkgreen}{rgb}{0.152941, 0.776471, 0.172549}
\definecolor{redpurple}{rgb}{0.835294, 0, 0.196078}
\definecolor{midblue}{rgb}{0, 0.592157, 0.662745}
\definecolor{clearpurple}{rgb}{0.67451, 0.0784314, 0.352941}
\definecolor{browngreen}{rgb}{0.333333, 0.313725, 0.145098}
\definecolor{darkestpurple}{rgb}{0.396078, 0.113725, 0.196078}
\definecolor{greypurple}{rgb}{0.294118, 0.219608, 0.298039}
\definecolor{darkturquoise}{rgb}{0, 0.239216, 0.298039}
\definecolor{darkbrown}{rgb}{0.305882, 0.211765, 0.160784}
\definecolor{midgreen}{rgb}{0.560784, 0.6, 0.243137}
\definecolor{darkred}{rgb}{0.576471, 0.152941, 0.172549}
\definecolor{darkpurple}{rgb}{0.313725, 0.027451, 0.470588}
\definecolor{darkestblue}{rgb}{0, 0.156863, 0.333333}
\definecolor{lightpurple}{rgb}{0.776471, 0.690196, 0.737255}
\definecolor{softgreen}{rgb}{0.733333, 0.772549, 0.572549}
\definecolor{offwhite}{rgb}{0.839216, 0.823529, 0.768627}


\title{Notes on \\ Pseudo-Boolean Implication Proofs \\ Version of \today}

\author{Randal E. Bryant}
\institute{
Computer Science Department \\
Carnegie Mellon University, Pittsburgh, PA, United States\\
\email{Randy.Bryant@cs.cmu.edu}
}


\begin{document}

\maketitle

This notes describes some ideas on converting unsatisfiability proofs
for pseudo-Boolean (PB) constraints into clausal proofs based on the
DRAT proof system.

\section{Notation}

This section gives brief definitions of pseudo-Boolean constraints and their properties.
A more extensive introduction is provided by Gocht in his PhD thesis [Gocht-Phd-2022].

Consider a set of Boolean variables $X = \{ x_1, x_2, \ldots, x_n \}$.  For $x_i \in X$, {\em literal}
$\lit_i$ can denote either $x_i$ or its negation, written $\obar{x}_i$.  We
write $\obar{\lit}_i$ to denote $\obar{x}_i$ when $\lit_i = x_i$ and $x_i$ when
$\lit_i = \obar{x}_i$.

A {\em pseudo-Boolean constraint} is a linear expression, viewing
Boolean variables as ranging over integer values $0$ and $1$.  That
is, for a set of indices $\indices_C \subseteq \{ 1, 2, \ldots, n \}$,
a constraint $C$ has the form
\begin{eqnarray}
\sum_{i \in \indices_C } a_i \lit_i & \# & b \label{eqn:pbconstraint}
\end{eqnarray}
where: 1) the relational operator $\#$ is $<$, $\leq$, $=$, $\geq$, or
$>$, and 2) the coefficients $a_i$, as well as the constant $b$, are
integers, with the coefficients being nonzero.
Constraints with relational operator $=$ are referred to as {\em equational constraints},
while others are referred to as {\em ordering constraints}.

Constraint $C$ denotes a Boolean function, written
$\func{C}$, mapping assignments to the set of variables $X$ to 1
(true) or 0 (false).  Constraints $C_1$ and $C_2$ are said to {\em
  equivalent} when $\func{C_1} = \func{C_2}$.
Constraint $C$ is said to be {\em infeasible} when $\func{C} = \bot$, i.e., it always evaluates $0$.
$C$ is said to be {\em trivial} when $\func{C} = \top$, i.e., it always evaluates to $1$.


As described in [Gocht-Phd-2022], the following are some properties of pseudo-Boolean constraints:
\begin{itemize}
\item Constraints with relational operators $<$, $\leq$, and $>$ can be
  converted to equivalent constraints with relational operator $\geq$.
\item
The logical negation of an ordering constraint $C$, written $\obar{C}$,
can also be expressed as a constraint.  That is, assume that $C$
has been converted to a form where it has relational operator $\geq$.
Then replacing $\geq$ by $<$ yields the negation of $C$.  
\item
A constraint $C$ with relational operator $=$ can be converted to the
pair of constraints $C_{\leq}$ and $C_{\geq}$, formed by replacing
$\#$ in (\ref{eqn:pbconstraint}) by $\leq$ and $\geq$, respectively.
These then satisfy $\func{C} = \func{C_{\leq}} \land \func{C_{\geq}}$.
\end{itemize}
We will generally
assume that constraints have relational operator $\geq$, since other
forms can be translated to it.  In some contexts, however, we will
maintain equational constraints 
intact, to avoid replacing it by two ordering constraints.


We consider two {\em normalized forms} for ordering constraints: A
{\em coefficient-normalized} constraint has only positive
coefficients.  A {\em variable-normalized} constraint has only
positive literals.  Converting between the two forms is
straightforward using the identity $\obar{x}_i = 1-x_i$.  In reasoning
about PB constraints, the two forms can be used interchangeably.
Typically, the coefficient-normalized form is more convenient when
viewing a PB constraint as a logical expression, while the
variable-normalized form is more convenient when viewing a constraint
as an arithmetic expression.  In this document, we focus on the logical
aspects, giving the general form of constraint $C$ as
\begin{eqnarray}
\sum_{i \in \indices_C} a_{i} \lit_{i} & \geq & b \label{eqn:coeff-normalized}
\end{eqnarray}
with each $a_{i} > 0$.

Ordering constraint $C$ in coefficient-normalized form is trivial if and only
if $b \leq 0$.  Similarly, $C$ is infeasible if and only if
$b > \sum_{i \in \indices_C} {a_{i}}$.  By contrast, testing feasiblilty or triviality
of an equational constraint is not straightforward, in that an instance of the
subset sum problem [Garey] can be directly encoded as an equational
constraint.

An {\em assignment} is a mapping $\assign : X' \rightarrow \{0,1\}$,
for some $X' \subseteq X$.  The assignment is {\em total} when $X' =
X$ and {\em partial} when $X' \subset X$.  Assignment $\assign$ can
also be viewed as a set of literals, where $x_i \in \assign$ when
$\assign(x_i) = 1$ and $\obar{x}_i \in \assign$ when $\assign(x_i) = 0$.
Finally, assignment $\assign$ can be viewed as the
pseudo-Boolean constraint $\sum_{\lit_i \in \assign} \lit_i \;\geq\;| \assign |$.
We use these three views interchangeably.  Note also for
assignments $\assign_1$ and $\assign_2$ over disjoint sets of
variables $X_1$ and $X_2$, their union is logically equivalent to
their conjunction: $\func{\assign_1 \cup \assign_2} = \func{\assign_1}
\land \func{\assign_2}$.

We shall make use of constraints representing the negations of
assignments.  That is, the negation of assignment $\assign$, written
$\nassign$ will is the constraint:
$\sum_{\lit_i \in \assign} \obar{\lit}_i \;\geq\; 1$.
Logically, this is equivalent to a clause in a CNF
representation.

We let $\assign(C)$ denote the constraint resulting when $C$ is
simplified according assignment $\assign$.  That is, assume $\assign$ is defined over a set of variables $X'$ and
partition the indices $i \in \indices_C$ into
three sets:
$I^{+}$, consisting of those indices $i$ such that $\lit_{i} \in \assign$,
$I^{-}$, consisting of those indices $i$ such that $\obar{\lit}_{i} \in \assign$,
and $I^{X}$ consisting of those indices $i$ such that $x_i \not \in X'$.
With this, $\assign(C)$ can be written
\begin{eqnarray}
\sum_{i \in I^{X}} a_{i} \lit_{i} & \geq & b - \sum_{i \in I^{+}} a_{i} \label{eqn:assigned}
\end{eqnarray}

A pseudo-Boolean {\em formula} $F$ is a set of pseudo-Boolean
constraints.  We say that $F$ is {\em satisfiable} when there is some
assignment $\assign$ that satisfies all of the constraints in $F$, and
{\em unsatisfiable} otherwise.  


\section{Pseudo-Boolean Implication Proofs}

A Pseudo-Boolean Implication Proof (PBIP) provides a systematic way to
prove that a PB formula $F$ is unsatisfiable.  It consists of a sequence of ordering constraints
\begin{displaymath}
  C_1, C_2, \ldots, C_m, C_{m+1}, \ldots, C_t
\end{displaymath}  
such that the first $m$ constraints are those of formula $F$, while the constraints $i$
with $i > m$ are implied by either one or two of the preceding constraints.
That is, for each step $i$ with $m < i \leq t$,
there is a set $H_i \subseteq \{C_1, C_2, \ldots, C_{i-1} \}$ such that $|H_i| \leq 2$ and
\begin{eqnarray}
\bigwedge_{C \in H_i} \func{C} & \imply & \func{C_i} \label{eqn:proofsequence}
\end{eqnarray}
The proof completes with an infeasible constraint $C_t$.
By the transitivity of implication, we have therefore proved that $F$ is not satisfiable.
We refer to the set $H_i$ as the {\em hints} for step $i$.

Unless ${\it P} = {\it NP}$, we cannot guarantee that a proof checker
can validate even a single step of a PBIP proof in polynomial time.
In particular, consider an equational constraint $C$ encoding an
instance of the subset sum problem, and let $C_{\leq}$ and $C_{\geq}$
denote its conversion into a pair of ordering constraints such that
$\func{C} = \func{C_{\leq}} \land \func{C_{\geq}}$.  Consider a PBIP
proof step to add the constraint $\obar{C}_{\leq}$ having the 
$C_{\geq}$ as the only hint.  Proving that
$\func{C_{\geq}} \imply \func{\obar{C}_{\leq}}$, requires proving that
$\func{C_{\leq}} \land \func{C_{\geq}} = \nil$, i.e., that $C$ is unsatisfiable.

On the other hand, checking the correctness of a PBIP proof can be
performed in {\em pseudo-polynomial} time, meaning that the complexity will
be bounded by a polynomially sized formula over the numeric values of
the integer parameters.  This can be done using binary decision
diagrams [BBH-2022].  In particular, an ordering constraint over $n$ variables in
coefficient-normalized form with constant $b$ will have a BDD
representation with at most $b \cdot n$ nodes.  For proof step where
the added clause and the hints all have constants less than or equal
to $b$, the number of BDD operations to validate the step will be
$O(b^{2} \cdot n)$ when there is a single hint and $O(b^{3} \cdot n)$
when there are two hints.  This complexity is polynomial in $b$, but
it would be exponential in the size of a binary representation of $b$.


\section{(Reverse) Unit Propagation}

Consider constraint $C$ in coefficient-normalized form.  Literal
$\lit$ is {\em unit propagated} by $C$ when the assignment $\assign = \{ \obar{\lit} \}$
causes the constraint $\assign(C)$ to become infeasible.
Observe that
a single constraint can unit propagate multiple literals.  For
example, $4 x_1 + 3 \obar{x}_2 + x_3 \geq 6$ unit propagates both
$x_1$ and $\obar{x}_2$.  For constraint $C$, we let $\uprop{C}$
denote the set of literals it unit propagates.  This set can also be
viewed as a partial assignment, and therefore also a constraint,
having the property that $\func{C} \imply \func{\uprop{C}}$.

For an ordering constraint $C$ in coefficent-normalized form (\ref{eqn:coeff-normalized}), detecting
which literals unit propagate is straightforward.  Let $A =
\sum_{i \in \indices_C} a_{i}$.  Then literal $\lit_{i}$ unit propagates if and only
if $A - a_{i} < b$, i.e., $a_{i} > A - b$.  For example, the constraint 
$4 x_1 + 3 \obar{x}_2 + x_3 \geq 6$ has $A = 7$ and $b=6$, yielding $A-b=1$.
This justifies the unit propagations of both $x_1$ and $\obar{x}_2$.
A constraint of the form $\lit \geq 1$ is the constraint representation of a unit clause---it will unit propagate $\lit$.

To perform unit propagation on constraint $C$ in the context of a
partial assignment $\alpha$, we can use (\ref{eqn:assigned}) to first
generate the constraint representation of $\assign(C)$.

The {\em reverse unit propagation} (RUP) proof rule [Gocht-Phd-2022] uses
unit propagation to prove that constraint $C$ can be added to 
formula $F$ while preserving its set of satisfying assignments.  It
operates by finding a sequence of constraints $C_0, C_1, \ldots, C_k$,
with $C_0 = \obar{C}$ and $C_i \in F$ for $1 \leq i \leq k$, and proving that this combination is unsatisfiable.

A successful RUP proof generates a sequence of assignments
$\assign_0, \assign_1, \ldots, \assign_{k-1}$ with $\assign_0 = \uprop{C_0}$, and
$\assign_i = \assign_{i-1} \cup \uprop{\assign_{i-1}(C_i)}$
for $1 < i < k$. [{\bf REB NOTE:} What about when $\assign_0$ is empty?]
That is, it extends the set of unit literals with those obtained by first applying the unit literals to $C_i$ and then performing unit propagation.
The final clause $C_k$ must then satisfy $\assign_{k-1}(C_k) = \nil$.

Given a RUP proof step, we can generate a sequence of PBIP proof steps that concludes with the addition of constraint $C$.
In particular, the final assignment in the RUP proof yielded the result $\func{\assign_{k-1}(C_k)} = \func{\assign_{k-1}} \land \func{C_k} \imply \nil$, and therefore
$\func{C_k} \imply \func{\nassign_{k-1}}$.
  Starting with $i = k$ and stepping downward to $i=1$, we can see
$\func{\assign_i} = \func{\assign_{i-1}} \land \func{C_i}$, but the previous step added
$\nassign_{i}$, and therefore we can add
the clause $\nassign_{i-1}$, with the preceding clause plus clause $C_i$ as hints.  Once we reach $i=1$, we have added the clause $\nassign_0$,
and we can therefore add clause $C$ with the preceding clause as its hint, since $\func{C_0} \imply \func{\assign_{0}}$, and therefore
$\func{\nassign_0} \imply  \func{\obar{C}_0} = \func{C}$.

Summarizing, the PBIP proof steps will be as follows:
\begin{center}
\begin{tabular}{ll}
\toprule
  \makebox[3cm]{Added Clause} & \makebox[2cm]{Hints} \\
  \midrule
  $\nassign_{k-1}$ & $C_k$  \\
  $\nassign_{k-2}$ & $C_{k-1}$, $\nassign_{k-1}$ \\
  $\cdots$ & $\cdots$ \\
  $\nassign_{i}$ & $C_{i+1}$, $\nassign_{i+1}$ \\
  $\cdots$ & $\cdots$ \\
  $\nassign_{0}$ & $C_{1}$, $\nassign_{1}$ \\
  $C$ & $\nassign_{0}$  \\
  \bottomrule
\end{tabular}
\end{center}

Note in these that the constraint representation of an
assignment or its inverse has a very simple form.  In particular,
if
$\assign = \{\lit_1, \lit_2, \ldots, \lit_k\}$,
then its contstraint
representation is $\lit_1 + \lit_2 + \cdots + \lit_k \geq k$.  On the
other hand, constraint $\nassign$ can be written as
$\obar{\lit}_1 + \obar{\lit}_2 + \cdots + \obar{\lit}_k \geq 1$.

\section{RUP Example}

As an example, consider the following three clauses:
\begin{center}
  \begin{tabular}{cllllll}
\toprule    
\makebox[1cm]{ID} & \multicolumn{6}{c}{Constraint} \\
\midrule
$C_1$ & \makebox[0.6cm][l]{$x_1$} & $+$ & \makebox[0.6cm][l]{$2 x_2$} & $+$ & \makebox[0.6cm][l]{$\obar{x}_3$} & \makebox[0.6cm][l]{$\geq 2$} \\
$C_2$ & $\obar{x}_1$ & $+$ & $\obar{x}_2$ & $+$ & $2 x_3$ & $\geq 2$ \\
$C_3$ & $x_1$ & $+$ & $2 \obar{x}_2$ & $+$ &  $3 \obar{x}_3$ & $\geq 3$ \\
\bottomrule
\end{tabular}
\end{center}
Our goal is to add the clause $C = 2 x_1 + x_2 + x_3 \geq 2$ via RUP using these clauses.  RUP proceeds as follows:
\begin{enumerate}
\item
We can see that $\obar{C} = 2 \obar{x}_1 + \obar{x}_2 + \obar{x}_3 \geq 3$, and this unit propagates assignment $\assign_0 = \obar{x}_1 \geq 1$.
\item
With this, constraint $C_1$ simplifies to $2 x_2 + \obar{x}_3 \geq 2$, and this unit propagates $x_2$, giving  $\assign_1 = \obar{x}_1 + x_2 \geq 2$.
\item
  Constraint $C_2$ simplifies to $2 x_3 \geq 1$, which unit propagates $x_3$, giving $\assign_2 = \obar{x}_1 + x_2 + x_3 \geq 3$.
\item
  Constraint $C_3$ simplifies to $ 0 \geq 3$, which is infeasible.
\end{enumerate}

This single application of RUP results in the following PBIP proof steps:
\begin{center}
  \begin{tabular}{cllllllll}
\toprule    
\makebox[1cm]{ID} & \multicolumn{6}{c}{Constraint} & \makebox[0.4cm]{} & \makebox[1cm]{Hints}\\
\midrule
$C_4$ & \makebox[0.6cm][l]{$x_1$} & $+$ & \makebox[0.6cm][l]{$\obar{x}_2$} & $+$ & \makebox[0.6cm][l]{$\obar{x}_3$} & \makebox[0.6cm][l]{$\geq 1$} && $C_3$ \\
$C_5$ & $x_1$ & $+$ & $\obar{x}_2$ & & & $\geq 1$ && $C_2$, $C_4$ \\
$C_6$ & $x_1$ &     &              & & & $\geq 1$ && $C_1$, $C_5$ \\
$C$   & $2x_1$ & $+$ & $x_2$ & $+$ & $x_3$ & $\geq 2$ && $C_6$ \\
\bottomrule
\end{tabular}
\end{center}




%%\bibliography{references}

\end{document}
